\documentclass[letterpaper]{article}

\usepackage{graphicx}
\usepackage{hyperref}
\usepackage[margin=1in]{geometry}
\graphicspath{ {./} }

\begin{document}

% Title -----------------------------------------------------------------------

\title{Reading 04: Diversity At Notre Dame}
\date{February 12, 2017}
\author{Jacob Beiter {\textless}jbeiter@nd.edu{\textgreater}}

\maketitle

% Overview --------------------------------------------------------------------

\section*{Overview}

\paragraph{}

In this reading assignment, I scraped demographic data about Notre Dame's Computer Science program from a csv that were used to create graphs, and wrote a report about the experience and the program's diversity using LaTeX.

% Methodology -----------------------------------------------------------------

\section*{Methodology}

\paragraph{}

I processed the demographics.csv data using a combination of curl, cut, grep, and wc. I used curl to fetch the data from the URL provided, and cut to trim only to the relevant column, using an arithmetic formula to convert from the year. With that column I used grep to search for the entries that matched what I was looking for, and then used wc to count the number.

% Analysis --------------------------------------------------------------------

\section*{Analysis}

\begin{center}
\begin{tabular}{ |r|c|c| }
 \hline
 \multicolumn{3}{|c|}{Gender Breakdown by Graduation Year}\\
 \hline
 Year & Female & Male\\
 \hline
 2013  & 14 & 49\\
 2014  & 12 & 44\\
 2015  & 16 & 58\\
 2016  & 19 & 60\\
 2017  & 26 & 65\\
 2018  & 36 & 90\\
 2019  & 51 & 97\\
 \hline
\end{tabular}
\end{center}

\begin{center}
\begin{tabular}{ |r|c|c|c|c|c|c|c| }
 \hline
 \multicolumn{8}{|c|}{Ethnicity Breakdown by Graduation Year}\\
 \hline
 Year & Caucasian & Oriental & Hispanic & African & Native & Multiple & Undeclared\\
 \hline
 2013  & 43 & 7 & 7 & 3 & 1 & 2 & 0\\
 2014  & 43 & 5 & 4 & 2 & 1 & 1 & 0\\
 2015  & 47 & 9 & 10 & 4 & 1 & 1 & 2\\
 2016  & 53 & 9 & 9 & 1 & 7 & 0 & 0\\
 2017  & 60 & 12 & 3 & 5 & 5 & 6 & 0\\
 2018  & 91 & 8 & 12 & 3 & 4 & 8 & 0\\
 2018  & 92 & 13 & 10 & 3 & 15 & 14 & 0\\
 \hline
\end{tabular}
\end{center}


\paragraph{}

The data in these tables and graphs show the gender and ethnic breakdown of the students in the Computer Science program at the University of Notre Dame. Enrollment of both females and non-white students has been increasing, but so also has the total enrollment.


\begin{figure}[t]
\centering
\includegraphics[scale=0.35]{gender}
\includegraphics[scale=0.35]{ethnic}
\end{figure}

\begin{enumerate}

\item{}The overall trend in the gender breakdown of the program is a positive one - over the last several years, the proportion of females in the Computer Science program is growing. In a field that is so heavily skewed in the male direction this is definitely a positive thing, but it remains to be seen how long this period of large growth in the program as a whole lasts, and where things stand once it is done.

\item{}Over the past years the absolute number of non-white students has been trending upwards. The proportion actually decreased between 2017 and 2018 because of the incredible growth in total enrollment of the program, but it is continuing to rise again with the 2019 class.

\end{enumerate}


\paragraph{}

% Discussion ------------------------------------------------------------------

\section*{Discussion}

\paragraph{}

\begin{itemize}

\item{}I think that it is extremely important to cultivate diversity in general, and certainly in computer science in specific - particularly because it is a field that has been so dominated by white males. Diversity as a principle is valuable because it can help foster different viewpoints and ideas, as well as make all feel welcome. Diversity as an idea is very important, both for Notre Dame and for the technology industry.

\item{}I think the Computer Science department here at Notre Dame has done good work to make itself welcoming to all, and we are trending in a good direction, but it simply takes time -- particularly when there has been such a large increase in total enrollment recently.

\item{}I personally haven't experienced any particular challenges thus far in the Computer Science department, but I'm not really of a demographic who would, if there were challenges being faced. As a white male, part of the demographic that has been so far dominating the technology industry, all that means is that we haven't erred too far on the side of diversity and downplayed the importance of people who don't increase diversity statistics.

\end{itemize}

\end{document}
